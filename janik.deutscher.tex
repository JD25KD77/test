\documentclass[11pt]{article}
%set the document class, this way latex knows the formatting of the document, the other popular one would be \documentclass{beamer} which gives slides. You can also add options to change font size etc.. Google as needed

\usepackage{amssymb, amsthm, amsmath}
\usepackage{comment, enumerate, enumitem}
\usepackage{xcolor, float}
\usepackage{tikz}
\usetikzlibrary{plotmarks}
\usepackage{pgfplots}

\theoremstyle{definition}
\newtheorem{theorem}{Theorem}
\newtheorem{acknowledgement}[theorem]{Acknowledgement}
\newtheorem{algorithm}[theorem]{Algorithm}
\newtheorem{axiom}[theorem]{Axiom}
\newtheorem{case}[theorem]{Case}
\newtheorem{claim}[theorem]{Claim}
\newtheorem{conclusion}[theorem]{Conclusion}
\newtheorem{condition}[theorem]{Condition}
\newtheorem{conjecture}[theorem]{Conjecture}
\newtheorem{corollary}[theorem]{Corollary}
\newtheorem{criterion}[theorem]{Criterion}
\newtheorem{definition}[theorem]{Definition}
\newtheorem{example}[theorem]{Example}
\newtheorem{exercise}[theorem]{Exercise}
\newtheorem{lemma}[theorem]{Lemma}
\newtheorem*{lemma*}{Lemma}
\newtheorem*{theorem*}{Theorem}

\newcommand\irregularcircle[2]{% radius, irregularity
	\pgfextra {\pgfmathsetmacro\len{(#1)+rand*(#2)}}
	+(0:\len pt)
	\foreach \a in {10,20,...,350}{
		\pgfextra {\pgfmathsetmacro\len{(#1)+rand*(#2)}}
		-- +(\a:\len pt)
	} -- cycle
}

\newtheorem{notation}[theorem]{Notation}
\newtheorem{problem}[theorem]{Problem}
\newtheorem{proposition}[theorem]{Proposition}
\newtheorem{remark}[theorem]{Remark}
\newtheorem{solution}[theorem]{Solution}
\newtheorem*{solution*}{Solution}
\newtheorem{summary}[theorem]{Summary}
\newtheorem*{hint}{Hint}
\pgfplotsset{compat=1.15}

% Adjust margins
\addtolength{\oddsidemargin}{-0.5in}
\addtolength{\evensidemargin}{-0.5in}
\addtolength{\textwidth}{1in}
\addtolength{\topmargin}{-1in}
\addtolength{\textheight}{1.0in}

\newcommand{\thedate}{\today}

\begin{document}
	%begins the document
    \begin{center}
        \textbf{\huge \scshape Effective Programming Practices for Economists - Assignment 1} \\ \vspace{1pt}
    \centering\textbf{\thedate} \\ 
    \item Janik Deutscher
    \end{center}

\section{Past experience doing research with a computer}  
Description 



\section{Exercises on Technology of Skill Formation}

\subsection*{(a) Six stylized facts}
\begin{itemize}
    \item Ability gaps between individualsand  across  socioeconomic  groups  open  up  at  early  ages,  for  both  cognitive  and  noncognitive skills.
    \item In  both  animal  and  human  species,  there  is  compelling  evidence  of  critical  and
    sensitive periods in the development of the child.
    \item Despite the low returns to interventions targeted toward disadvantaged adolescents,the empirical literature shows high economic returns for remedial investments in young disadvantaged children.
    \item If early investment in disadvantaged children is not followed up by later investment, its effect at later ages is lessened.
    \item The effects of credit constraints on a child’s adult outcomes depend on the age at which they bind for the child’s family.
    \item Socioemotional (noncognitive) skills foster cognitive skills and are an important product of successful families and successful interventions in disadvantaged families.
\end{itemize}

\subsection*{(b) Description of model}
Description of model 

\subsection*{(c) Explanations of relevant concepts}

    \subsubsection*{(i) Critical period}
    If one stage alone is effective in producing a skill or ability, it is called a "critical period" for that skill. 
    \subsubsection*{(ii) Sensitive period}
    The stages that are more efficient in producing certain skills are called "sensitive periods" for the acquisition of those skills. 
    \subsubsection*{(iii) Dynamic complementarity}
    Skills produced at one stage raise the productivity of investment at subsequent stages.  

   
\subsection*{(d) Dynamic complementarity with linear production function}
The production function is defined as $\theta_{t+1} = f_t(h,\theta_t,I_t)$, which can be written as $\theta_{t+1}=\alpha*h+\beta*\theta_t +\gamma*I_t$ for some $\alpha, \beta, \gamma \in \mathbb{R}_+$ if linearity is assumed. The second partial derivative can then be written as follows: 
\begin{equation}
    \frac{\partial^2 f_t(h,\theta_t,I_t)}{\partial \theta \partial I'_t} = \frac{\partial^2 (\alpha*h+\beta*\theta_t +\gamma*I_t)}{\partial \theta \partial I'_t} = \frac{\partial \beta}{\partial I'_t} = 0    
\end{equation}

\noindent The assumption made when choosing a linear production function is thus that dynamic complementarity does not exist. 

\subsection*{(e) Ignoring differences between skills and test scores}
Explanation

\end{document}